\vspace{-2pt}
\section{Related Work}
\label{sec:related}

%Architecture
\newtext{
A larger number of architecture researches have been done for manually designing one common platform for homogeneous applications with repeated computation patterns, e.g. Minerva\cite{reagen2016minerva}, Eyeriss\cite{chen2018eyeriss} for Deep Neural Network applications with repeated convolution computing. 
To efficiently execute the application with heterogeneous kernels, monolithic ACC has been implemented at the beginning, e.g.\cite{tamitani1992encoder}\cite{chawla2016image}. 
However, the monolithic application-specific ACC has low flexibility and is not efficient for other applications. To support many applications in one architecture, ACC-rich platform\cite{cong2014accelerator}\cite{tabkhi2014function} is promising, where many small ACCs can be composed to accelerate different large kernels (or even applications).

There are some other high-performance architecture works supporting many applications. 
Reconfigurable computing, such as \cite{wildermann2011operational}, aims to support multiple applications one at a time using FPGA, however, it relies on functional and structural similarities across reconfiguration cycles.
CPU associated with GPU\cite{grasso2014energy} is also a high-computing choice. However, GPU is too general and exceeds power and area limitation in embedded design\cite{maghazeh2013general}, especially in edge computing.
}

%Some promising architectures for many-application platforms have been proposed (e.g., \cite{tabkhi2014function, nowatzki2017domain}), but many-application DSE and platform evaluation have less been tackled. 

%Design Automation
\newtext{
Current design automation mainly focuses on one isolate application \cite{SCE}\cite{Daedalus}\cite{gruttner2011challenges}. The design flow covers two steps: platform architecture allocation and application-specific mapping (included binding and scheduling).
There are huge researches about mapping different applications on the existing architecture\cite{marwedel2011mapping}\cite{quan2013scenario}. However, platform allocation for many applications is less touched. 
Some ideas about how to consider many applications can be extracted from platform-based design\cite{PBD}\cite{lukasiewycz2009combined}. It uses statistical information to design the platform interconnections for many applications. However, its ACC allocation is either from engineering experiences or application-specific \cite{PBDmapping}\cite{gladigau2010system}, which requires large NRE or narrows the final architecture flexibility from many applications to a single application.
}


%algorithm
\newtext{
In design algorithm. heuristic methods are widely used to deal with the vast and complex design space.
The heuristics methods are comprised of an algorithm to explore design space and a fast evaluation/estimation to judge the performance of chosen platforms.
Some examples are genetic algorithms (GA)~\cite{quan2014towards}, simulated annealing~\cite{liang2013hardware}, tabu search~\cite{wu2013efficient}, and greedy algorithms~\cite{tang2015hardware}.
These existing algorithms focus on allocating/mapping a platform for a single application because there is less existing platform evaluation that considers many applications.
The only early example of a many-application algorithm is Domain Score Selection (DSS)~\cite{zhang2018ds} which proposes a greedy approach for identifying similar function kernels for hardware acceleration, but it uses an unfair platform evaluation, comparing average throughput improvement, wherein a small number of high-performance applications dominate the evaluation. 
}
%Due to the vast and complex design space, heuristic methods are widely used in DSEs.
%The heuristics methods are comprised of an algorithm to explore design space and a fast evaluation/estimation to judge the performance of chosen platforms.
%Some examples are genetic algorithms (GA)~\cite{quan2014towards}, simulated annealing~\cite{liang2013hardware}, tabu search~\cite{wu2013efficient}, and greedy algorithms~\cite{tang2015hardware}.
%These existing DSEs focus on allocating a platform for a single application, because there is less existing platform evaluation that considers many applications. 

%Some ideas about how to consider many applications can be extracted from platform-based computing which uses statistical information to design the platform template and application-specific mappings~\cite{graf2014multi, gladigau2010system}. However, more specialization for a wider set of applications is needed. Reconfigurable computing, such as \cite{wildermann2011operational}, aims to support multiple applications one at a time, however, it relies on functional and structural similarities across reconfiguration cycles.

%Some promising architectures for many-application platforms have been proposed (e.g., \cite{tabkhi2014function, nowatzki2017domain}), but many-application DSE and platform evaluation have not yet been tackled. 
%An early example of a many-application DSE is Domain Score Selection (DSS)~\cite{zhang2018ds} which proposes a greedy approach for identifying similar function kernels for hardware acceleration, but it uses an unfair platform evaluation, comparing average throughput improvement, wherein a small number of high-performance applications dominate the evaluation. 