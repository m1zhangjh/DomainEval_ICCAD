\begin{abstract}

% Jinghan Abstract
%Domain-specific computing is a promising solution to bridge the flexibility/efficiency gap for a broader set of applications. Streaming applications within a domain, such as video analytics, software-defined radio, and radar, benefit from domain specialization due to functional and structural similarities. To aid their design, new evaluation methodologies and tools are needed with a broader scope of application set instead of individual applications.

%This paper introduces a novel domain-specific platform evaluation methodology for domain-specific computing with a focus on streaming applications. Key contributions are: (1) A methodology flow to evaluate the efficiency of platform for domain applications, (2) relative efficiency to have a fair view for all domain applications, (3) relative efficiency aggregation of domain applications to quantitatively judge domain platform and (4) using different fair evaluations to guide domain-specific design space exploration (DS-DSE).


% Gunar suggest Abstract
%Using heterogeneous accelerator-rich (ACC-rich) platforms which combine processor cores with specialized HW accelerators (ACCs), is one main approach to high-performance low-power computing.
%ACCs can be shared across many applications, when they have functional and structural similarities, especially in streaming applications, such as video analytics, software-defined radio, and radar.
%However, many applications on the same platform are only considered as individual mapping problem to a fixed platform. Current Design Space Exploration (DSE) of platform allocation mostly focuses on single applications.
%To increase the scope of DSE of platform allocation to consider many applications, efficient traversal and fair evaluation of platforms across applications are needed.


Heterogeneous accelerator-rich (ACC-rich) platforms combining general-purpose cores and specialized HW accelerators (ACCs) promise high-performance and low-power streaming application deployments, e.g. for video analytics, software-defined radio, and radar. 
In order to recover NRE, a unified platform for a set of applications is desirable. When applications 
have functional and structural similarities, they can benefit from common ACCs. One challenge is to identify 
the most beneficial set of ACCs to deploy those applications on a unified platform.
However, current allocation strategies mostly produce a dedicated platform for one application in isolation. 
Automating the allocation of a unified platform for many applications requires 
broadening the scope to many applications, 
efficient design space traversal and 
a fair evaluation across diverse applications.


This paper introduces UPA, a Unified ACC-rich Platform Allocation methodology for sets of data flow applications.
Key contributions in UPA are:
(1) a genetic algorithm (GA) guided by a fair and efficient evaluation to allocate one unified platform for many applications, 
(2) a definition of relative efficiency to fairly compare the improvement of applications on a platform, 
and (3) evaluating aggregation of the performance of many applications to quantitatively compare the efficiency of platforms.
This paper demonstrates UPA's benefits using OpenVX applications. 
UPA platform with 12 ACCs has a 4.59x higher average efficiency improvement than considering one application in isolation for platform allocation. Moreover, it shows a 1.41x higher efficiency improvement over a related approach based on Dynamic Score Selection (DSS)~\cite{zhang2018ds}.
%
In addition, UPA platform enables more applications (67.5\% of the OpenVX apps) to be efficiently deployed (improving efficiency $\geq$ 3), while DSS platform only targets 42.5\%.

%\vspace{-4pt}

\end{abstract}