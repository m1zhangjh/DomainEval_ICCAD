\begin{abstract}

% Jinghan Abstract
%Domain-specific computing is a promising solution to bridge the flexibility/efficiency gap for a broader set of applications. Streaming applications within a domain, such as video analytics, software-defined radio, and radar, benefit from domain specialization due to functional and structural similarities. To aid their design, new evaluation methodologies and tools are needed with a broader scope of application set instead of individual applications.

%This paper introduces a novel domain-specific platform evaluation methodology for domain-specific computing with a focus on streaming applications. Key contributions are: (1) A methodology flow to evaluate the efficiency of platform for domain applications, (2) relative efficiency to have a fair view for all domain applications, (3) relative efficiency aggregation of domain applications to quantitatively judge domain platform and (4) using different fair evaluations to guide domain-specific design space exploration (DS-DSE).


% Gunar suggest Abstract
Using heterogeneous accelerator-rich (ACC-rich) platforms which combine processor cores with specialized HW accelerators (ACCs), is one main approach to high-performance low-power computing.
ACCs can be shared across many applications, when they have functional and structural similarities, especially in streaming applications, such as video analytics, software-defined radio, and radar.
However, many applications on the same platform are only considered as individual mapping problem to a fixed platform. Current Design Space Exploration (DSE) of platform allocation mostly focuses on single applications.
To increase the scope of DSE of platform allocation to consider many applications, efficient traversal and fair evaluation of platforms across applications are needed.

This paper introduces Many Applications ACC-Rich (MAAR) DSE with kernel granularity.
Key contributions in MAAR are: (1) a genetic algorithm (GA) guided by \newtext{a fair and efficient evaluation} to allocate a platform for many applications, (2) a definition of relative efficiency to fairly compare the improvement of applications on a platform, and (3) evaluating aggregation of the performance of many applications to quantitatively compare the efficiency of platforms. 
This paper validates MAAR DSE by using it to allocate a platform for OpenVX applications. Compared with single application DSE (1appDSE) and previous domain DSE Dynamic Score Selection (DSS~\cite{zhang2018ds}), MAAR has 4.01 and 1.39 times the average efficiency improvement respectively. With a budget of 12 ACCs, MAAR enables more applications (67.5\% of applications) to achieve high efficiency (efficiency improvement $\geq$ 3) than 1appDSE (15\%) and DSS(42.5\%).

%\vspace{-4pt}

\end{abstract}