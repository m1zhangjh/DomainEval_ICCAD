\vspace{-2pt}
\section{Conclusion}
\label{sec:conclusion}
This paper introduced a novel Design Space Exploration (DSE) approach for Many Applications Accelerator-Rich (MAAR) platforms. The aim of MAAR DSE is designing next generation of Accelerator-Rich platforms to address emerging computationally diverse markets such as computer vision and software-defined radio, by broadening the scope of DSE from a single application in isolation to many applications. The proposed MAAR DSE employs elitist GA for fast traversal and a fair evaluation methodology for balancing the focus and accelerator allocation across all applications. Our results on OpenVX applications demonstrate that MAAR DSE obtains 4.01 and 1.39 times more relative efficiency improvement than 1AppDSE and DSS respectively. With a 12 ACCs budget, MAAR platform supports more applications (67.5\%) with high efficiency (3x) which is much higher then existing solutions. 

%while 1appDSE and DSS only support 15\% and 42.5\% applications respectively.

%We choose $Alog$ as the best, because it is most fair for low-efficiency applications, and it does not to need to normalize application efficiency.

%The MAAR-DSE proposed fair aggregation solutions to judge platform fitness qualitatively to balance efficiency and accelerator selection across all applications