\vspace{-2pt}
\section{Conclusion}
\label{sec:conclusion}

This paper introduced a novel Unified Platform Allocation (UPA) supporting many applications. The aim of UPA is designing next generation of accelerator-rich platforms to address emerging computationally diverse markets such as computer vision and software-defined radio, by broadening the scope of allocation from a single application in isolation to many applications. The proposed UPA employs elitist GA for fast traversal and a fair evaluation methodology for balancing the focus and accelerator allocation across all applications. With a 12 ACCs budget, our results on OpenVX applications demonstrate that UPA obtains 4.59 and 1.41 times more relative efficiency improvement than FDP and DSS respectively. UPA platform supports more applications (67.5\%) with high efficiency (3x) which is much higher than existing solutions.
\newtext{
UPA platform with a 2 ACCs budget (allocated \emph{Custom Convlution} and \emph{Canny Edge Detector} kernels) achieves 67\% the average efficiency of ODP, which uses 25 unique ACCs.
}
%while 1AppDSE and DSS only support 15\% and 42.5\% applications respectively.

%We choose $Alog$ as the best, because it is most fair for low-efficiency applications, and it does not to need to normalize application efficiency.

%The MAAR-DSE proposed fair aggregation solutions to judge platform fitness qualitatively to balance efficiency and accelerator selection across all applications