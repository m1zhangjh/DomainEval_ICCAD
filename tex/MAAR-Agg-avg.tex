\subsubsection{Average Aggregation}

To judge the fitness of a platform for all applications, the intuitive method is to aggregate the overall efficiency increase or loss across all applications.
In \figref{fig:effSW}, a MAAR platform with a more blue area (or a less grey area) is better, because it has a more overall increase (or a less loss). 
To achieve the best MAAR platform $p_{X}$, DSE should maximize the blue area ($Area_{b}$) or minimize the grey area ($Area_{g}$). 
Assuming the length of total number of applications in y-axis is equal to 1, 
$Area_{b}$ is equal to $\sum_{i=0}^{\#a} ( rEFF_{SW}(a_{i}, p_{X}) -  rEFF_{SW}(a_{i}, p_{SW}) ) / \#a $, and $Area_{g}$ is $\sum_{i=0}^{\#a} ( rEFF_{SW}(a_{i}, p_{OOP}) -  rEFF_{SW}(a_{i}, p_{X}) ) / \#a $.
After simplification, both maximizing $Area_b$ and minimizing $Area_g$ become equivalent to maximizing $\sum_{i=0}^{\#a} rEFF_{SW}(a_{i}, p_{X}) / \#a$, which is the average efficiency improvement of platform $p_{X}$.  

Both methods are equivalent because of the SW and OOP platforms are fixed for a set of applications, so the summed SW and OOP efficiency areas are constant. 
Therefore, maximizing the efficiency difference between platform $p_{X}$ and SW, or minimizing the difference between $p_{X}$ and OOP leads to the same result, which is to maximize the average efficiency improvement of $p_{X}$. 
The average of $rEEF_{SW}$ ($A \mhyphen SW$) across all applications becomes the first MAAR DSE fitness function, which is represented in the first line of Eq.~\eqref{eq:avg}.

\vspace{-8pt}
\begin{equation}
\begin{split}
	A\mhyphen SW (p_{X}) &= \sum_{i=0}^{\#a} rEFF_{SW}(a_{i}, p_{X}) / \#a \\
	A\mhyphen OOP (p_{X}) &= \sum_{i=0}^{\#a} rEFF_{OOP}(a_{i}, p_{X}) / \#a
\label{eq:avg}
\end{split}
\end{equation}

Similarly in \figref{fig:effOOP}, maximizing the difference in $rEFF_{OOP}$ between MAAR platform $p_{X}$ and SW, and minimizing the difference between $p_{X}$ and OOP are both equivalent to maximize the average efficiency achievement $rEFF_{OOP}$, as displayed in the second line of Eq.~\eqref{eq:avg}. $A \mhyphen OOP$ is the second proposed fitness function.