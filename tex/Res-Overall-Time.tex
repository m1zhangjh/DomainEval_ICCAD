\subsubsection{Scalability Analysis}
\label{subsubsec:overall-time}

%\begin{figure}[h]
%\vspace{-8pt}
%	\centering
%		\subfloat[OpenVX: Diff \#ACCs] {\includegraphics[width=.48\linewidth]{fig/timeACCs_all.pdf}\label{fig:timeACCs_all}}
%		\hfill
%		\subfloat[Synthetic: Diff \#Apps] {\includegraphics[width=.48\linewidth]{fig/timeApps_all.pdf}\label{fig:timeApps_all}}
%	\vspace{-8pt}
%	\caption{Exploration Time}
%	\label{fig:exTime_all}
%\end{figure}


\begin{figure}[ht]
  \centering
  \begin{minipage}[t]{0.235\textwidth}
    \includegraphics[width=.95\textwidth]{fig/timeACCs_all.pdf}
    \caption{Exploration Time}
    \label{fig:timeACCs_all}
  \end{minipage}%
  \hfill
  \begin{minipage}[t]{0.235\textwidth}
    \includegraphics[width=.95\textwidth]{fig/oopHW.pdf}
    \caption{Unique ACCs Used}
	\label{fig:oopHW}
  \end{minipage}
\end{figure}

\newtext{
\figref{fig:timeACCs_all} shows the exploration time of different allocations running on Intel i5-3450 with 3.10GHz.
%\figref{fig:timeACCs_all} demonstrates the exploration time for OpenVX applications with an increasing ACC budget.
Thanks to high-level allocation (kernel) and fast evaluation (analytic model), all allocations are very fast.
With the highest design complex ACCs=19, UPA exploration only takes 475.56 seconds.
From ACCs=1 to ACCs=19, both DSS and UPA exploration time increases, because the platform design space gets bigger with the increasing number of ACCs. 
%From ACCs=1 to ACCs=19, DSS's exploration time increases from 0.084 seconds to 0.245 seconds, and UPA is from 85.10 seconds to 475.56 seconds. 
DSS is much faster than UPA, since DSS is only a greedy ACC selection algorithm and UPA needs to explore and compare different ACC allocations. 
FDP has an almost constant exploration time because it is an exhaustive search for each individual application, and the number of applications is fixed.
}

%With the number of ACCs increasing, the individual application design space does not increase much, because the number of unique kernels (ACC candidates) is only 2-9, while there are 35 kernels across many applications. 

%\figref{fig:timeApps_all} shows the exploration time versus an increasing number of applications using synthetically generated applications. The total number of unique kernels in all application sets are 35, and the allocating platform(s) has a budget ACCs=19. 
%The DSS exploration time is almost constant around 0.245 seconds, because it selects ACC using profiled characteristics from applications, and the number of applications does not impact on the exploration \cite{zhang2018ds}.
%With increasing applications from 40 to 100, UPA exploration time increases from 475.56 seconds to 971.26 seconds, because it contains an increasing number of individual applications evaluation for each platform candidate.
%Similarly, FDP exploration time increases significant from 131.65 seconds to 477.75 seconds, because of the more number of applications, the of more exhaustive search.   
