\subsubsection{Log Area Aggregation}

In previous average aggregation, high relative efficiency applications have more effect on fitness. The value difference between high and low relative efficiency across applications is still a little significant, especially in $rEFF_{SW}$. To shrink the difference among applications and pay more attention to low-efficiency applications, a new aggregation with logarithm is proposed.
Instead of maximizing/minimizing the relative efficiency difference between MAAR platform and SW/OOP, the new method ($Alog$) first takes the logarithm of the efficiency to shrink the difference in relative efficiency among applications, then maximizes/minimizes the resulting logarithmic area.

\begingroup\makeatletter\def\f@size{6.9}\check@mathfonts
\vspace{-8pt}
\begin{equation}
\begin{split}
	Alog\mhyphen SW(p_{X})
	&= \sum_{i=0}^{\#a} ( \log rEFF_{SW}(a_{i}, p_{X}) - \log rEFF_{SW}(a_{i}, p_{SW}) ) / \#a \\
	&= \sum_{i=0}^{\#a} ( \log rEFF_{SW}(a_{i}, p_{X}) ) / \#a 
	%&= \sum_{i=0}^{\#a} ( \log EFF(a_{i}, p_{X}) - \log EFF(a_{i}, p_{SW}) - \log 1 ) / \#a \\
	%&= \sum_{i=0}^{\#a} ( \log EFF(a_{i}, p_{X}) ) / \#a - \sum_{i=0}^{\#a} ( \log EFF(a_i, p_{SW}) ) / \#a
\label{eq:logSW}
\end{split}
\end{equation}
\endgroup

\begingroup\makeatletter\def\f@size{6.9}\check@mathfonts
\vspace{-8pt}
\begin{equation}
\begin{split}
	Alog\mhyphen OOP(p_{X})
	&= \sum_{i=0}^{\#a} ( \log rEFF_{OOP}(a_{i}, p_{X}) - \log rEFF_{OOP}(a_{i}, p_{SW}) ) / \#a \\
	&= \sum_{i=0}^{\#a} ( \log \frac{EFF(a_{i}, p_{X})}{EFF(a_{i}, p_{OOP})} - \log \frac{EFF(a_{i}, p_{SW})}{EFF(a_{i}, p_{OOP})} ) / \#a \\
	&= \sum_{i=0}^{\#a} ( \log rEFF_{SW}(a_{i}, p_{X}) ) / \#a
	%&= \sum_{i=0}^{\#a} ( \log EFF(a_{i}, p_{X}) ) / \#a - \sum_{i=0}^{\#a} ( \log EFF(a_i, p_{SW}) ) / \#a
\label{eq:logOOP}
\end{split}
\end{equation}
\endgroup


Eq.~\eqref{eq:logSW} shows the the log area of the efficiency improvement $Alog\mhyphen SW$. 
\newtext{
Since $\log rEFF_{SW}(a_i, p_{SW}) = 0$ for all $a_i$, $Alog\mhyphen SW$ becomes maximizing $\sum_{i=0}^{\#a} \log rEFF_{SW}(a_{i}, p_{X}) / \#a$. In Eq.~\eqref{eq:logOOP}, the log area of achievement $Alog\mhyphen OOP$ becomes the same as $Alog\mhyphen SW$, after relative efficiency transformation and logarithm simplification.
}

\begingroup\makeatletter\def\f@size{6.9}\check@mathfonts
\vspace{-8pt}
\begin{equation}
\begin{split}
	Alog(p_{X}) &\equiv Alog\mhyphen SW(p_{X}) \equiv Alog\mhyphen OOP(p_{X}) \\
	&= \sum_{i=0}^{\#a} ( \log rEFF_{SW}(a_{i}, p_{X}) ) / \#a  \\
	&= \sum_{i=0}^{\#a} ( \log \frac{EFF(a_{i}, p_{X})}{EFF(a_{i}, p_{SW})}) / \#a \\
	&= \sum_{i=0}^{\#a} ( \log EFF(a_{i}, p_{X}) ) / \#a - \sum_{i=0}^{\#a} ( \log EFF(a_i, p_{SW}) ) / \#a
\label{eq:logDetial}
\end{split}
\end{equation}
\endgroup

\newtext{
Next in Eq.~\eqref{eq:logDetial}, $Alog$ could transforms into two components, the logarithmic area of platform $p_{X}$ efficiency subtracts the logarithmic area of software efficiency. 
}
Since the efficiency of an application in software $EFF(a_i, p_{SW})$ is constant for all applications, it does not affect which platform $Alog\mhyphen SW$ is maximized for. Therefore, the normalization factor becomes unimportant in log area, so $Alog$ becomes equivalent to the log area of efficiency as shown in Eq.~\eqref{eq:log}. The fact that the normalization factor ``disapears'' in $Alog$ suggests that taking the logarithm of the efficiency of an application removes its dependence on the application's potential for efficiency without the need for a specific normalization. 


\begingroup\makeatletter\def\f@size{9}\check@mathfonts
\vspace{-8pt}
\begin{equation}
\begin{split}
	Alog(p_{X}) = \sum_{i=0}^{\#a} ( \log EFF(a_{i}, p_{X}) ) / \#a	
\end{split}
\label{eq:log}
\end{equation}
\endgroup